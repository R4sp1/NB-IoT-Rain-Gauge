\chapter*{Úvod}
\phantomsection
\addcontentsline{toc}{chapter}{Úvod}
\par Při výzkumu možností měření objemu spadlých srážek pomocí komerčních mikrovlnných spojů je jedním z problémů nemožnost ověření výstupních informací. Meteorologické srážkoměrné stanice jsou zpravidla umístěna pouze na místech, které jsou meteorologicky zajímavé a slouží hlavně k tvorbě předpovědí počasí. Jejich data také často nejsou veřejně dostupná. Vzniká zde tedy potřeba po nízkonákladových srážkoměrných stanicích, které budou kombinovat nízkou spotřebu pro možnosti běhu měření i bez dostupné energetické sítě a relativně přesné měření objemu spadlých srážek.
\par V této práci se zabýváme řešením tohoto problému. V prví části zjišťujeme možnosti automatizovaného měření objemu spadlých srážek a porovnáváme technologie pro přenos naměřených dat pomocí low-power wide-area sítí (LPWAN). V druhé části využijeme nasbírané informace a zabýváme se návrhem nízkoenergetického zařízení, jež by s dostatečnou přesností měřilo objem spadlých srážek a pro přenos naměřených dat využilo sítě NB-IoT.bO přenos dat v síti NB-IoT se bude starat modul Quectel BC660K-GL, zatímco řízení celého zařízení bude mít na starost mikrokontroler Espressif ESP32.
\par Cílem práce je návrh prototypu zařízení, které by ve své finální podobě sloužilo k podpoře probíhajícího výzkumu měření objemu srážek pomocí komerčních mikrovlnných spojů, který probíhá na půdě Vysokého učení technického v Brně.


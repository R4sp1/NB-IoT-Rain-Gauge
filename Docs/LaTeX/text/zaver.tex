\chapter*{Závěr}
\phantomsection
\addcontentsline{toc}{chapter}{Závěr}

\par V této práci jsme prozkoumali možnosti automatizovaného měření objemu spadlých srážek a představili návrh nízkoenergetického zařízení pro sběr naměřených dat. Využili jsme člunkového srážkoměru, jež kombinuje dostatečnou přesnost měření s minimální energetickou náročností a je tak vhodný pro použití v zařízení, které se bude vyskytovat mimo dosah energetické sítě. Pro přenos naměřených dat jsme využili NB-IoT modulu Quectel BC660K-GL, který podporuje standardy LTE cat. NB1 a LTE cat. NB2. Pro řízení zařízení jsme zvolili mikrokontroler Espressif ESP32.
\par Dalším krokem bude testování a optimalizování zařízení pro reálné nasazení. Budeme se také zabývat porovnáním člunkových srážkoměrů v němž budeme především pozorovat rozdíl mezi levným (v řádu stovek Kč) a dražším typem (v řádu tisíců Kč). Hlavním cílem bude také optimalizace komunikace mezi mikrokontrolerem Espressif ESP32 a NB-IoT modulem Quectel BC660K-GL a optimalizace samotného chování NB-IoT modulu.
